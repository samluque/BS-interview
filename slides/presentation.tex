\documentclass[aspectratio=169,11pt]{beamer}

% Theme
\usetheme{metropolis}
\usepackage{appendixnumberbeamer}

% Packages
\usepackage{booktabs}
\usepackage{graphicx}
\usepackage{xcolor}
\usepackage{amsmath}

% Colors
\definecolor{profit}{RGB}{34, 197, 94}
\definecolor{cost}{RGB}{220, 38, 38}
\definecolor{primary}{RGB}{37, 99, 235}
\definecolor{muted}{RGB}{100, 116, 139}

% Metadata
\title{Ad Performance Case Study}
\subtitle{Facebook App Install Campaign Analysis}
\author{Samuel Luque}
\date{\today}
\institute{}

\begin{document}

% =============================================================================
% TITLE
% =============================================================================
\maketitle

% =============================================================================
% OPENING HOOK
% =============================================================================
\begin{frame}{The Headline}
\begin{center}
\Large
\textbf{This campaign was highly profitable}\\[1em]
\Huge
\textcolor{profit}{\textbf{+138\% ROI}}\\[1em]
\normalsize
\textcolor{muted}{But that's not what the dashboard told us.}
\end{center}

\pause
\vspace{1em}
\begin{columns}
\column{0.5\textwidth}
\centering
\textbf{Dashboard said:}\\
\$0.02 profit/install
\column{0.5\textwidth}
\centering
\textbf{Reality:}\\
\$0.05 profit/install
\end{columns}

\vspace{1em}
\centering
\textcolor{muted}{\small More than double.}
\end{frame}

% -----------------------------------------------------------------------------
\begin{frame}{Three Questions}
\begin{enumerate}
\item \textbf{How many installs did the campaign actually create?}\\
\textcolor{muted}{\small (Spoiler: far more than attribution shows)}
\vspace{0.5em}
\pause
\item \textbf{What's the true value of each user?}\\
\textcolor{muted}{\small (Spoiler: more than day-0 metrics suggest)}
\vspace{0.5em}
\pause
\item \textbf{Can we make this even better?}\\
\textcolor{muted}{\small (Spoiler: yes, by shifting budget by weekday)}
\end{enumerate}
\end{frame}

% =============================================================================
% THE SITUATION
% =============================================================================
\section{The Situation}

\begin{frame}{Campaign Overview}
\begin{columns}
\column{0.5\textwidth}
\textbf{The app:}
\begin{itemize}
\item Free iOS app
\item Revenue from in-app purchases \& ads
\item Market: Brazil
\end{itemize}

\vspace{1em}
\textbf{The campaign:}
\begin{itemize}
\item Platform: Facebook
\item Duration: Oct 7 -- Dec 31, 2014
\item Daily budget cap: \$90
\item Single creative, broad targeting
\end{itemize}

\column{0.5\textwidth}
\textbf{Data sources:}
\begin{enumerate}
\item \textbf{Campaign data} (Facebook)\\
\small Spend, attributed installs, day-0 revenue
\normalsize
\vspace{0.5em}
\item \textbf{App-wide data}\\
\small Total downloads \& revenue (all users)
\end{enumerate}

\vspace{1em}
\textcolor{primary}{\textbf{Key insight:}} The second dataset shows what attribution \textit{doesn't} capture.
\end{columns}
\end{frame}

% =============================================================================
% QUESTION 1: HOW MANY INSTALLS?
% =============================================================================
\section{Question 1: How many installs?}

\begin{frame}{The Halo Effect}
\begin{center}
\includegraphics[width=0.75\textwidth]{figures/chart1_halo.png}
\end{center}
\end{frame}

% -----------------------------------------------------------------------------
\begin{frame}{The Halo Effect: Key Numbers}
\begin{center}
\begin{tabular}{lcc}
\toprule
& \textbf{Before Campaign} & \textbf{During Campaign} \\
\midrule
Non-attributed downloads/day & 13 & 855 \\
\bottomrule
\end{tabular}
\end{center}

\pause
\vspace{1em}
\begin{center}
\Huge\textcolor{profit}{\textbf{$60\times$ increase}}
\end{center}

\vspace{1em}
\begin{center}
\textcolor{muted}{These are real users the campaign brought in, but Facebook didn't credit.}
\end{center}
\end{frame}

% -----------------------------------------------------------------------------
\begin{frame}{Two Ways to Calculate CPI}
\begin{center}
\includegraphics[width=0.95\textwidth]{figures/chart2_cpi.png}
\end{center}
\end{frame}

% -----------------------------------------------------------------------------
\begin{frame}{CPI: Key Insight}
\begin{columns}
\column{0.5\textwidth}
\textbf{CPI\textsubscript{paid}} (Facebook reports)\\
\Large \$0.07\\
\normalsize
\textcolor{muted}{Spend $\div$ Attributed installs}

\pause
\column{0.5\textwidth}
\textbf{CPI\textsubscript{all}} (includes halo)\\
\Large \textcolor{primary}{\textbf{\$0.04}}\\
\normalsize
\textcolor{muted}{Spend $\div$ Total installs}
\end{columns}

\pause
\vspace{1.5em}
\begin{center}
\textcolor{profit}{\textbf{43\% lower acquisition cost}} when we include halo installs.\\[1em]
\textbf{Working assumption:} CPI\textsubscript{all} is our true acquisition cost.\\[0.5em]
\textcolor{muted}{\small (The $60\times$ jump is too large to be coincidence, but we'll validate with a geo holdout.)}
\end{center}
\end{frame}

% =============================================================================
% QUESTION 2: TRUE VALUE?
% =============================================================================
\section{Question 2: True value per user?}

\begin{frame}{The LTV Problem}
\textbf{What we have:}
\begin{itemize}
\item Day-0 revenue per install: \$0.017 (from Facebook)
\item Daily totals: revenue \& installs
\end{itemize}

\vspace{1em}
\textbf{What we need:}
\begin{itemize}
\item 30-day value per user
\item Payback period
\end{itemize}

\vspace{1em}
\textbf{The challenge:} No user-level data.

\vspace{1em}
\textbf{The approach:} Use \textit{deconvolution} --- work backward from daily revenue patterns to estimate revenue-by-age.
\end{frame}

% -----------------------------------------------------------------------------
\begin{frame}{LTV Curve vs Acquisition Cost}
\begin{center}
\includegraphics[width=0.9\textwidth]{figures/chart3_ltv.png}
\end{center}
\end{frame}

% -----------------------------------------------------------------------------
\begin{frame}{LTV: Key Numbers}
\begin{center}
\begin{tabular}{lc}
\toprule
\textbf{Metric} & \textbf{Value} \\
\midrule
30-day LTV & \$0.092 \\
Acquisition cost (CPI\textsubscript{all}) & \$0.039 \\
\midrule
\textbf{30-day margin/install} & \textcolor{profit}{\textbf{\$0.053}} \\
\textbf{Payback} & \textcolor{profit}{\textbf{Day 0}} \\
\textbf{30-day ROI} & \textcolor{profit}{\textbf{+138\%}} \\
\bottomrule
\end{tabular}
\end{center}

\pause
\vspace{1em}
\begin{center}
\textcolor{muted}{63\% of value captured on day 0; 87\% by day 7.}
\end{center}
\end{frame}

% -----------------------------------------------------------------------------
\begin{frame}{Daily Profitability: Dashboard vs Reality}
\begin{center}
\includegraphics[width=0.95\textwidth]{figures/chart4_profit.png}
\end{center}
\end{frame}

% -----------------------------------------------------------------------------
\begin{frame}{The Punchline}
\begin{center}
\Large
The dashboard showed \textbf{\$0.02} profit per install.\\[1em]
\pause
The real 30-day profit is \textbf{\textcolor{profit}{\$0.05}} per install.\\[2em]
\normalsize
\textcolor{muted}{Day-0 metrics understate true economics by more than half.}
\end{center}

\pause
\vspace{1em}
\textbf{December drift:} Performance weakened in December.\\
\textcolor{muted}{Likely cause: creative fatigue, audience saturation.}\\
\textcolor{muted}{Action: refresh creatives, monitor cohort quality.}
\end{frame}

% =============================================================================
% QUESTION 3: CAN WE DO BETTER?
% =============================================================================
\section{Question 3: Can we do better?}

\begin{frame}{Weekday Seasonality}
\textbf{Observation:} Some weekdays look more efficient than others.

\vspace{1em}
\textbf{Challenge:} Campaign drifts over time (December $<$ October).

\vspace{1em}
\textbf{Solution:} Compare weekdays \textit{within the same week}.
\begin{itemize}
\item Is Tuesday better than the average day in that week?
\item This removes drift, isolates weekday effects.
\end{itemize}

\vspace{1em}
\textbf{Goal:} Turn weekday patterns into a daily cap schedule.
\end{frame}

% -----------------------------------------------------------------------------
\begin{frame}{Weekday ROI: Within-Week Comparison}
\begin{center}
\includegraphics[width=0.85\textwidth]{figures/chart5_weekday.png}
\end{center}
\end{frame}

% -----------------------------------------------------------------------------
\begin{frame}{Weekday Caps: The Recommendation}
\begin{center}
\begin{tabular}{lccc}
\toprule
\textbf{Day} & \textbf{Relative ROI} & \textbf{Multiplier} & \textbf{Daily Cap} \\
\midrule
Tuesday & \textcolor{profit}{$+15\%$} & $1.15\times$ & \$104 \\
Monday & \textcolor{profit}{$+5\%$} & $1.05\times$ & \$94 \\
Sunday & \textcolor{profit}{$+4\%$} & $1.04\times$ & \$94 \\
Wednesday & $-2\%$ & $0.98\times$ & \$88 \\
Saturday & $-5\%$ & $0.96\times$ & \$86 \\
Thursday & \textcolor{cost}{$-8\%$} & $0.92\times$ & \$83 \\
Friday & \textcolor{cost}{$-10\%$} & $0.91\times$ & \$81 \\
\bottomrule
\end{tabular}
\end{center}

\vspace{1em}
\begin{center}
\textbf{Weekly spend stays the same:} $7 \times \$90 = \$630$
\end{center}
\end{frame}

% -----------------------------------------------------------------------------
\begin{frame}{How to Validate}
\textbf{A/B test design:}
\begin{itemize}
\item \textbf{Week A:} Flat caps (\$90/day)
\item \textbf{Week B:} Weekday-weighted caps (table above)
\item Alternate weeks, compare CPI, installs/\$, revenue/\$
\end{itemize}

\vspace{1em}
\textbf{Why test?}
\begin{itemize}
\item Shifting budget can change ROI (diminishing returns)
\item Multipliers should be tuned based on results
\end{itemize}

\vspace{1em}
\textcolor{muted}{Low-risk: we learn something either way.}
\end{frame}

% =============================================================================
% LIMITATIONS & NEXT STEPS
% =============================================================================
\section{Limitations \& Next Steps}

\begin{frame}{Limitations}
\begin{enumerate}
\item \textbf{No randomized control group}\\
\textcolor{muted}{Halo effect based on step-change, not a true experiment.}
\vspace{0.5em}
\item \textbf{Aggregate data only}\\
\textcolor{muted}{LTV inferred from totals, not tracked per user.}
\vspace{0.5em}
\item \textbf{Cohort quality may vary}\\
\textcolor{muted}{Single LTV estimate may not hold for all cohorts.}
\vspace{0.5em}
\item \textbf{Single creative, single market}\\
\textcolor{muted}{Results may not generalize.}
\end{enumerate}
\end{frame}

% -----------------------------------------------------------------------------
\begin{frame}{Next Steps}
\textbf{Validate:}
\begin{itemize}
\item Geo holdout test: ads on in some cities, off in others
\item Measure lift in installs and revenue
\end{itemize}

\vspace{0.5em}
\textbf{Operate:}
\begin{itemize}
\item Test weekday caps (flat vs weighted)
\item Refresh creatives to address December drift
\end{itemize}
\end{frame}

\begin{frame}{Next Steps (II)}
\textbf{Measure better:}
\begin{itemize}
\item Track LTV by cohort and source (paid vs non-attributed)
\item Monitor D0/D7/D30 revenue weekly
\end{itemize}

\vspace{0.5em}
\textbf{Scale:}
\begin{itemize}
\item Budget step test to map diminishing returns
\item Then: increase spend where ROI holds
\end{itemize}
\end{frame}

% =============================================================================
% SUMMARY
% =============================================================================
\begin{frame}{Summary}
\begin{center}
\Large
\textbf{This campaign works.}
\end{center}

\vspace{1em}
\begin{columns}
\column{0.33\textwidth}
\centering
\textbf{CPI\textsubscript{all}}\\
\Large \$0.039
\column{0.33\textwidth}
\centering
\textbf{LTV\textsubscript{30}}\\
\Large \$0.092
\column{0.33\textwidth}
\centering
\textbf{ROI}\\
\Large \textcolor{profit}{+138\%}
\end{columns}

\pause
\vspace{1.5em}
\begin{center}
\normalsize
\textbf{Key insight:} The dashboard understated profit by half.\\[0.5em]
\textbf{Opportunity:} Shift budget to better weekdays (+15\% on Tuesdays).\\[0.5em]
\textbf{Next:} Validate with geo holdout, test weekday caps, refresh creatives.
\end{center}
\end{frame}

% -----------------------------------------------------------------------------
\begin{frame}[standout]
Questions?
\end{frame}

\end{document}
